\documentclass{article}
\usepackage[utf8]{inputenc}

\usepackage{geometry}
\usepackage{float}
\usepackage{siunitx}

\usepackage{multirow}
\usepackage{subfig}
\usepackage{rotating}

\newgeometry{margin=2.5cm}
\newcommand\todo[1]{\textbf{TODO: #1}}
\newcommand\done[1]{}


% clustering
% climatology
% tropical
% wind shear
% profiles
% climate model
%\title{Clustering wind shear profiles to produce a climatology of wind shear in a climate model}
%\title{Producing a climatology of wind shear by clustering wind shear profiles in a climate model}
\title{A climatology of wind shear produced by clustering wind profiles from a climate model}
% \title{Deriving a climatology of tropical wind shear profiles from a climate model}
\author{Mark Muetzelfeldt}

\begin{document}

\maketitle

% I am writing this as a *self-contained* piece of work, i.e. it is not directly linked to my thesis.
% This means that the primary motivation is to identify shear conditions that may be conducive for org, as opposed
% to working out the RWPs themselves.

\section{Introduction}

% Why wind shear profiles? -> org
% Why problem is hard - large param space -> need for simplification.
% Links to previous work:
% - climatologies of wind
% - climatologies of wind shear
% - climatologies of org
% - clustering of vars in weather/climate models.
% Outline of rest of paper:
% Methods - model and clustering procedure
% Results - brief overview of each RWP + global distn of RWPs and seasonal cycles
% Discussion and Conclusion - links to prev studies of org, links to case studies of org, future work.

% Set the scene.
Vertical wind shear is [an important/a key] factor in the organization of convection in the tropics. From theory, wind shear has been shown to provide the conditions under which squall lines can form \todo{cite RKW, moncrieff}, through the interaction of convectively generated cold pools and the environmental shear. Many case studies have highlighted the presence of wind shear when the convective cloud field has been organized \todo{cite many}. Review studies into the organization of convection into squall lines, Mesoscale Convective Systems (MCSs) and Mesoscale Convective Complexes (MCCs) have emphasized the role of wind shear in the formation of these types of organization \todo{cite fritsch2001mesoscale, houze2004mesoscale}. Developing a climatology of shear in a climate model can therefore serve as a basis for working out when and where conditions favourable for shear-induced organization of convection will occur. 

% Motivation and what can this be used for, and justification for the analysis done.
%\undone{Consider moving "Knowing where..." to start; it is the primary motivation after all}
%After talking to BP, switched back to putting analysis *in this paper* front and foremost. Using RWPs to perform further experiments will be left to future work.
% Knowing where the conditions for organization occur in a climate model is an important first step towards including a representation of some aspects of shear-induced organization in the model's representation of convection. This could be done by modifying the model's convection parametrization scheme to represent some aspects of organization, as has been suggested by \todo{cites}, and a method for doing this is discussed as future work. However, first it is judicious to compare the climatology of shear in the model to observed distributions of organized convection, such as \todo{cite}. This helps to build confidence that the hypothesized link between shear in the model and organized convection holds, and that the model is producing shear where it should. Further, comparing the shear profiles generated by the model with observed shear profiles from case studies of convective organization also provides a check that the that the model is producing realistic shear profiles. Characteristic shear profiles associated with specific regional features, e.g. the African Easterly Jet, should also be identifiable in the climatology of shear in the model. These comparisons are done in this study, and form the basis of the Results section.
Knowing where the conditions for organization occur in a climate model, it is then possible to compare the climatology of shear in the model to observed distributions of organized convection, such as \todo{cite}. This helps to build confidence that the hypothesized link between shear in the model and organized convection holds, and that the model is producing shear where it should. Further, comparing the shear profiles generated by the model with observed shear profiles from case studies of convective organization also provides a check that the that the model is producing realistic shear profiles. Characteristic shear profiles associated with specific regional features, e.g. the African Easterly Jet, should also be identifiable in the climatology of shear in the model. These comparisons are done in this study, and form the basis of the Results section.

% high-level overview of procedure.
A climatology is necessarily a simplification that relies on representing the statistical nature of some variable over many years. The question this study sets out to answer is: ``how can a climatology of a variable with a large parameter space be created?''. In this case, the variable is wind shear, although the method should be applicable to other such variables. \todo{active/passive voice?} [A method is used/We have designed a method] that relies on simplifying the representation of this variable as much as possible, while still retaining the essential features that make the climatology interesting. We make assumptions about which similarities and differences between the various shear profiles are important, as set out and justified in the \todo{} section. Through application of our clustering procedure, we reduce the space of all shear profiles down to 10 Representative Wind Profiles (RWPs) that effectively span the space. The 10 RWPs can then be analysed in turn, e.g. allowing us to see where a specific RWP occurs in space and time.

% Wind shear climatologies.
In a previous study, \todo{cite aiyyer2006climatology} looked into the climatology of vertical wind shear over the tropical Atlantic. They defined wind shear as a difference in wind speed between \SIrange{200}{850}{hPa}, and developed a climatology over \SI{46}{years}. Their approach to dealing with the large parameter space was to simplify it dramatically, treating shear as the difference between wind speeds at two levels. However, their motivation was primarily to look for conditions under which tropical cyclones could form, and whether this could be linked to ENSO, and so is quite different from the focus of this study. \todo{cite houchi2010comparison} developed a global climatology of wind shear profiles, from the surface up to \SI{30}{km}. In comparison with radiosonde data, they show that the ECMWF model under study produces wind shears that are 2.5 to 3 times smaller than the observed profiles. \todo{They do this to assess the kinds of shear profiles that will be returned from a weather satellite, ADM-Aeolus, that will measure these profiles using \todo{what?}.}

% Conv org climatologies.
Although relatively few climatologies of wind shear are available, more have been produced for the organization of convection \todo{cites}.
Some of these provide geographical distributions of where various types of organized convective systems are likely to occur \todo{cite}. These tend to be based on satellite observations though, and so do not relate the organization to the wind shear, which in many cases may be responsible for creating the organization. However, comparison of the shear climatology produced here with these climatologies increases confidence both in the fact that the climate model is producing the conditions for organization in the right locations, and that the organization is being influenced by the shear profile. \todo{say more about these}.

% Case studies.
Many case studies have looked into specific events where the organization of convection was active, e.g. \todo{cites}, or regions where there is a typical mode of organization, e.g. \todo{cites}. Some of these provide hodographs or wind profiles; these can be compared to the RWPs produced here to look for similarities between them from certain regions.

% Paper structure.
The rest of this study is structured as follows. In the Methods section, we provide information about the Global Circulation Model (GCM) used to generate a suitable dataset of wind profiles, as well as detailing the clustering procedure used to turn these profiles into a set of Representative Wind Profiles (RWPs). In the Results section, we look at the individual RWPs and link these to similar shear profiles seen in case studies, and analyse the spatial and temporal distribution of the RWPs. In the Discussion section, we link our results to previous studies of organization, and outline some ideas for future work. In the Conclusion, a brief summary is given.

\section{Methods}

\subsection{Climate model}

%\begin{itemize}
%    \item MO UM vn10.9
%    \item GA7.0
%    \item N96 + horizontal resolution at equator (around \SI{209}{km})
%    \item N96 has a total of $145 \times 192$ grid points
%    \item Run for \SI{1}{month}
%    \item Run on ARCHER using u-au197 (copy of u-ar683, standard GA7.0 suite), archived to RDF (\todo{paths})
%    \item $u$, $v$ and $w$ output on 7 pressure levels: \SIlist{950;900;850;800;700;600;500}{hPa}
%    \item CAPE also output
%    \item diagnostics output every \SI{6}{hr}, which gives a total of (4 * 30 * 145 * 192) \si{3340800} profiles
%\end{itemize}
%
The GCM that is used is the United Kingdom Met Office's Unified Model (UM), version 10.9. It is run using the standard Global Atmosphere 7.0 (GA7.0) settings, as described in depth in \todo{cite walters2018met}. The interested reader is directed to that paper for the full details. It is an atmosphere only model, using prescribed sea surface temperatures. It uses a version of the Gregory-Rowntree convection scheme \todo{cite gregory1990}. Here, it is run with an N96 resolution, which has a resolution of \SI{209}{km} at the equator. It is run for five years, from September 1988 to August 1993 using a \SI{360}{day} calendar. Running for five years allows us to sample inter-annual variation. East-west ($u$) and north-south ($v$) winds are output on 20 pressure levels from \SI{1000}{hPa} to \SI{50}{hPa} with a resolution of \SI{50}{hPa}, and are output every \SI{6}{hours}. Convective Available Potential Energy (CAPE) is also output every \SI{6}{hours}; it is calculated by the model's convection scheme and output as a diagnostic field. Profiles are only considered in the Tropics, defined as being between \ang{30}N and \ang{30}S.

\subsection{Generating the Representative Wind Profiles}

\subsubsection{Overview of clustering procedure}
\done{too low-level for an introduction -  move to methods}
\todo{make this fit in to a methods section}
% Problem in how to produce climatology, how we go about doing it.
The GCM used here has 70 vertical levels, with an east-west and north-south component of the wind at each level. This leads to a large parameter space for shear profiles. The problem of deriving a climatology of these profiles then becomes one of choosing how to reduce the complexity of the parameter space, whilst still maintaining the essential features that link a given group of profiles together. To do this, we have made some simplifying assumptions about what aspects of the profiles will provide useful information about shear-induced organization. 

First, we only consider wind values over the depth of the troposphere, from \SI{1000}{hPa} to \SI{50}{hPa} in steps of \SI{50}{hPa}. Each shear profile, or sample, then has 40 dimensions.  
\done{add in todo for lower trop being more important for org.}
\todo{perhaps reference chen upper trop important here?}
We also recognize that the lower troposphere is more important for the organization of convection by weighting the contribution to the analysis from the lower troposphere, up to \SI{500}{hPa}, more highly. This is necessary to stop the higher-level jets dominating the the analysis and clustering procedure. 
\todo{This can be justified}  \todo{cite RKW, moncreiff1992}, as well as . 
Second, as we are concerned with tropical convection, we are performing the analysis in a region where the variation in the Coriolis effect will be small. We therefore choose to neglect the relative rotation of the shear profiles.

% PCA and K-Means
The dimensionality of the problem can be further reduced by using Principal Component Analysis (PCA) to extract principal components that capture most of the variance of the samples with fewer dimensions. Then the samples can then be grouped together using a clustering algorithm. Clustering is a form of unsupervised machine learning. It groups similar samples in a dataset together, based on how close they are to each other. In this study we use the K-Means Clustering Algorithm (KMCA), as it provides a simple and efficient way of clustering like samples together. Once the samples have been clustered, the median of each cluster is referred to as a Representative Wind Profile (RWP). The clustering is done entirely on the values from one grid-column, so this technique is a data-driven way of grouping together like grid-columns. It could be performed on any set of values from a grid-column, and a similar method was used by \todo{cite} to group together land grid-cells in a climate model based on which plant types were present. Similar techniques have also been used in analysis of satellite imagery to classify plant types \todo{cite}.

The clustering procedure is shown schematically in Fig. \todo{}, and the details for each of the steps follow.

\begin{figure}[htp!]%
    \caption{\todo{fig} Schematic of clustering procedure.}%
    \label{fig:clustering_procedure}%
\end{figure}

\subsubsection{Filtering}

Profiles are filtered on two criteria. The filters reduce the number of profiles, excluding ones that either are not likely to produce convective activity or are not profiles with large amounts of shear. The filters are applied independently, so their order of application makes no difference. The filters in use are as follows:

\begin{enumerate}
    \item exclude grid points where CAPE \textless \SI{100}{J.kg^{-1}}; 
    \item exclude grid points where the maximum shear in the profile is less than the 75\textsuperscript{th} percentile. The shear is calculated between each pressure level, and only the shear values up to a value of \SI{500}{hPa} are taken into account to favour the lower troposphere.
\end{enumerate} 

\todo{say how many profiles are left after each of the filters are applied.}

\begin{figure}[h!]
    \centering
    \caption{\todo{fig} Heatmap of grid-columns that have been filtered based on CAPE and maximum shear.}
    \label{fig:filtered_heatmap}
\end{figure}


\subsubsection{Pre-processing}
\label{sec:preproc}

It is necessary to pre-process the data before performing the principal component analysis and the clustering. Two types of pre-processing are applied:

\paragraph{Normalize magnitude:}

this is to ensure that differences in the profiles at each pressure level each contribute the same amount to the distance measure used by the KMCA \todo{mention in Clustering procedure}. This involves normalizing each profile by the maximum magnitude of the wind at each pressure level (i.e. $\sqrt{u^2 + v^2}$).

To favour the lower troposphere, an extra factor is applied to the normalization above \SI{500}{hPa}: the normalized values above this height are reduced by a factor of four. This has the effect of reducing the contribution of differences in the upper troposphere when applying the clustering.

% It is true that the shape of the profiles is changed by the normalization; some concerns were raised by RP and PC that this could change the shear values for the profiles. It does change this shape, but it keeps the relative differences between the different profiles. Therefore, when the KMCA is applied, it will still cluster like profiles together, based on its distance metric.

\paragraph{Rotation:}

this is applied to treat profiles that share rotational symmetry in the same way. It is done by using the wind vector at \SI{850}{hPa} to define a rotation angle. All profiles are then rotated so that this angle is zero, i.e. all the profiles are aligned in the same direction at \SI{850}{hPa}. A small number of profiles (4\% \todo{more precise}) have a wind magnitude less than \SI{1}{m.s^{-1}} at \SI{850}{hPa}, these profiles are included but it should be noted that they may influence results. After applying this normalization, $u'$ refers to the direction aligned with the wind vector at \SI{850}{hPa}, and $v'$ is the orthogonal direction to this.

The reason for applying this normalization is that in the Tropics, it makes little difference whether a profile has e.g. unidirectional shear in the zonal or meridional direction. A similar point stands for all profiles, such as profiles that veer/back with height. This argument applies exactly only at a given latitude; there will be some differences between a given profile at one latitude versus another at a different latitude due to differences in the Coriolis effect. For the purposes of this analysis, we are neglecting these. 

\subsubsection{Principal component analysis}

\done{explain what is used for before saying how it works}

PCA is used to reduce the number of dimensions of each sample, by projecting each sample onto a truncated set of principal components.
PCA is a process that finds orthogonal, unit length principal components of a dataset that are linear combinations of the original axes. It does this in such a way that the first principal component accounts for the largest possible variance in the underlying dataset. The second principal component is orthogonal to the first, and accounts for the as much of the remaining variance as possible in the dataset, and so on. PCA can be used to reduce the number of dimensions of a dataset in a way the keeps the maximum possible variance for a given number of dimensions, by truncating the number of principal components used. 
%In Meteorology, PCA is also known as Empirical Orthogonal Functions (EOF), when it is applied to geographical datasets to look for the dominant modes of variability, e.g. the North Atlantic Oscillation.

The algorithm for PCA works by centring the dataset on the mean for each of its dimensions, and then calculating the covariance matrix for this dataset. The principal components can then be taken by finding the eigenvectors of the covariance matrix, with the first component corresponding to the eigenvector with the highest eigenvalue, and so on. 

The number of dimensions of each sample in the original dataset is 40 (20 pressure levels for $u$ and $v$). We have chosen to keep as many principal components that are required to explain 90\% of the variance, which for this dataset is 7. These 7 principal components are shown in Fig. \ref{fig:pca_components}. \todo{say something about these PCs}.

\begin{figure}[htp!]%
    \caption{\todo{fig} The first 7 principal components.}%
    \label{fig:pca_components}%
\end{figure}

% It is also insightful to look at the profiles that are produced by reprojecting the original profile along only these first 4 principal components. Some examples showing high and low fidelity to the original profiles are shown in Fig. \ref{fig:pca_reprojections}. \todo{some basic stats of how good keeping the first 4 components is? Or is this implicit in the fact that I have chosen to keep 90\% of the variance.}

%\begin{figure}[htp!]%
%    \centering
%    \subfloat[]{{\includegraphics[width=7cm]{figs/PCA_RED_True_cape-shear_magrot_725164_-4_nclust-11_prof-3452.png} }}%
%    \qquad
%    \subfloat[]{{\includegraphics[width=7cm]{figs/PCA_RED_True_cape-shear_magrot_725164_-4_nclust-11_prof-7767} }}%
%    \qquad
%    \subfloat[]{{\includegraphics[width=7cm]{figs/PCA_RED_True_cape-shear_magrot_725164_-4_nclust-11_prof-5178} }}%
%    \qquad
%    \subfloat[]{{\includegraphics[width=7cm]{figs/PCA_RED_True_cape-shear_magrot_725164_-4_nclust-11_prof-12082} }}%
%    \qquad
%    \caption{4 profiles that have been re-projected using the first 4 principal components.}%
%    \label{fig:pca_components}%
%\end{figure}

\subsubsection{K-means clustering}

The K-Means Clustering Algorithm (KMCA) splits a number of samples into clusters based on how similar the samples are to other samples. It does this in a way which minimizes the within cluster variance. The algorithm used here is LLoyd's algorithm, which is computationally efficient but not guaranteed to find a global minimum for the within cluster variance. It starts by randomly assigning samples to clusters. Then it calculates the mean of each cluster, and then it re-assigns samples to new clusters based in which of the cluster means each sample is nearest to \todo{define distance metric somewhere}. This last step is repeated until there is very little movement (less than some predefined threshold) in the cluster means, when the algorithm terminates.

The number of clusters to use is not \textit{a priori} obvious. We have the competing requirements that we want few enough RWPs that we can analyse where each one comes from without being overloaded with data, and we want each RWP to be as representative as possible of its cluster of profiles. One method for working out a suitable number of clusters is known as the elbow method. This involves plotting the sum of the variances of each cluster, the score, against the number of clusters used. This will necessarily decrease in magnitude as the number of clusters increases, however, if there is a change in the rate of change in the gradient (a so-called `elbow'), then this indicates that using more clusters is reducing the score by a lower amount, and thus indicates a suitable number of clusters. Figure \ref{fig:kmeans_scores} shows the scores for 5-20 clusters. Although there is not a large elbow in this figure, there is a recognizable kink at 8 clusters and 10 clusters. We pick 10 clusters as a pragmatic number of clusters to use, being large enough to span the wind profile space, and small enough that sensible analysis of each cluster is possible.

\todo{mention the random seeds producing the same clusters.}

\begin{figure}[h!]
    \centering
    \includegraphics[width=400px]{figs/KMEANS_SCORES_True_cape-shear_magrot.png}
    \caption{}
    \label{fig:kmeans_scores}
\end{figure}

\newpage
\section{Results}

\subsection{Representative wind profiles}
\subsection{Geographical distribution of RWPs}
\subsection{Seasonal variation of RWPs}

\begin{figure}[h]
    \centering
    \includegraphics[width=400px]{figs/PROFILES_GEOG_LOC_True_cape-shear_magrot_391137_-4_nclust-11.png}
    \caption{}
    \label{fig:RWPs}
\end{figure}

\section{Discussion}

\todo{How representative are the RWPs? Show nearest/furthest profiles here.}

\todo{Discuss choice of n_clust=10 here?}

\todo{I need to mention somewhere about the link to parametrization of org in a GCM.}

\todo{Highlight general applicability of work beyond shear.}

\todo{Future work:}
\todo{Run high res expts with each RWP}
\todo{Do diurnal analysis and see if it matches org of conv distn}
\todo{Run on reanalysis dataset}

\section{Conclusion}

\newpage
\section*{Appendix}

\subsection*{Background}

This is a short document that gives some history and technical details of the clustering work that I did from September 2017 to February 2018. The idea was that I needed to be able to link the grid-column state of a parametrized model to the subgrid shear-induced organizational state. The broad method I was trying was to investigate the wind profiles (WPs) produced by the model, and cluster these into Representative Wind Profiles (RWPs). This would reduce the `wind profile space' from a space that was too large to analyse to say 20 RWPs. Each of these RWPs could then be used to drive an RCE experiment, and then I could use the information from these experiments (such as cloud lifetimes) to feedback into the convection parametrization scheme. (This is future work.)

Initial results from last year were promising, using an sklearn K-Means Clustering Algorithm (KMCA), working on the standard output $u$ and $v$ fields at 6 pressure levels (\todo{what were they?}) collected every \SI{6}{hr} over the course of \SI{1}{month} of simulation. I only considered profiles between \ang{30}N and \ang{30}S over all longitudes. I filtered out profiles having a CAPE of less than \SI{500}{J.kg^{-1}}, or profiles with $w$ \textgreater  \SI{0}{m.s^{-1}}. I found the principal components of the profiles (using PCA), and used this to reduce the number of dimensions from 12 to around 4 (which explained over 90 \% of the variance). This was done as an advised precursor to using the KMCA, as an attempt to reduce the impact of the so-called `curse of dimensionality', and additionally there is interesting information in the shape of the principal components. I made it so that I could quickly experiment with e.g. running with or without filtering, PCA etc. Choosing 20 clusters for the KMCA, and plotting all the WPs for each RWP, it was believable that each cluster did capture some information about the underlying WP sample space, and that the RWPs could be of interest. Initially I was plotting every profile ($u$ and $v$) for each cluster, which made it hard to get a sense of what the mean profile looked like, although did not hide any information away. I have switched to plotting the mean, the mean +/- 1 standard deviation, and the 25\% and 75\% lines. N.B. it is easy to produce figures/results for each combination of options you can think of; it is harder to analyse them all. 

However, the profiles went too high into the atmosphere (\todo{what pressure?}), and because of this the clustering was being dominated by the high-level winds. To improve this, the profiles now only go up to \SI{500}{hPa}. Also, we worked out that we probably did not care about the relative orientation of the different WPs, and additionally we perhaps did not care about the sense of rotation (clockwise or anti-clockwise). I have now normalized for the orientation of the profiles, but not for the sense of rotation. I can also now filter on multiple attributes, and am now filtering on both CAPE and the maximum wind shear of each profile has to be greater than the lowest 25 \% of profiles. Additionally, I am normalizing the profiles on their magnitude at each pressure level separately (see Section \ref{sec:normalization} for more details and justification).


\end{document}
